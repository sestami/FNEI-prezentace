\documentclass[10pt,a4paper]{article}
\usepackage[utf8]{inputenc}
\usepackage[english,czech]{babel}
\usepackage{amsmath}
\usepackage{amsfonts}

\usepackage{empheq}
\usepackage[most]{tcolorbox}

\newtcbox{\mymath}[1][]{%
    nobeforeafter, math upper, tcbox raise base,
    enhanced, colframe=blue!30!black,
    colback=blue!30, boxrule=1pt,
#1}

\usepackage{booktabs}
\usepackage{longtable}
\usepackage{amssymb}
\usepackage{siunitx}
\usepackage[version=4]{mhchem}
\usepackage{textcomp} %for \textdegree
\usepackage{graphicx}
\usepackage{float}
\usepackage{subcaption}
\usepackage[left=2cm,right=2cm,top=4cm,bottom=4cm]{geometry}
\author{Michal Šesták}
\title{Infrazvuk poznámky}
\usepackage[unicode, pdfauthor={Michal Šesták}, pdftitle={Microbaroms}, pdfsubject={FNEI},colorlinks=true, linkcolor=red,
urlcolor=blue, citecolor=blue]{hyperref}
\LTcapwidth=\textwidth
\renewcommand{\arraystretch}{1.0}
\begin{document}

\renewcommand{\figurename}{Fig.}
\renewcommand{\tablename}{Tab.}
\maketitle
\section{Microbaroms}
Isolated traveling, ocean surface gravity waves radiate only evanescent acoustic waves,[7] and don't generate microbaroms.[16] Microbaroms are generated by nonlinear interactions of ocean surface waves traveling in nearly opposite directions with similar frequencies in the lee of a storm,[17] which produce the required standing wave conditions,[16] also known as the clapotis.[18] When the ocean storm is a tropical cyclone, the microbaroms are not produced near the eye wall where wind speeds are greatest, but originate from the trailing edge of the storm where the storm generated waves interact with the ambient ocean swells.[19]

Microbaroms may also be produced by standing waves created between two storms,[17] or when an ocean swell is reflected at the shore.[20] Waves with approximately 10-second periods are abundant in the open oceans, and correspond to the observed 0.2 Hz infrasonic spectral peak of microbaroms, because microbaroms exhibit frequencies twice that of the individual ocean waves.[17] Studies have shown that the coupling produces propagating atmospheric waves only when non-linear terms are considered.[9]

Microbaroms are a form of persistent low-level atmospheric infrasound,[21] generally between 0.1 and 0.5 Hz, that may be detected as coherent energy bursts or as a continuous oscillation.[11] When the plane wave arrivals from a microbarom source are analyzed from a phased array of closely spaced microbarographs, the source azimuth is found to point toward the low-pressure center of the originating storm.[22] When the waves are received at multiple distant sites from the same source, triangulation can confirm the source is near the center of an ocean storm.[4]

Microbaroms that propagate up to the lower thermosphere may be carried in an atmospheric waveguide,[23] refracted back toward the surface from below 120 km and above 150 km altitudes,[17][24] or dissipated at altitudes between 110 and 140 km.[25] They may also be trapped near the surface in the lower troposphere by planetary boundary layer effects and surface winds, or they may by ducted in the stratosphere by upper level winds and returned to the surface through refraction, diffraction or scattering.[26] These tropospheric and stratospheric ducts are only generated along the dominant wind directions,[24] may vary by time of day and season,[26] and will not return the sound rays to the ground when the upper winds are light.[17]

The angle of incidence of the microbarom ray determines which of these propagation modes it experiences. Rays directed vertically toward the zenith are dissipated in the thermosphere, and are a significant source of heating in that layer of the upper atmosphere.[25] At mid latitudes in typical summer conditions, rays between approximately 30 and 60 degrees from the vertical are reflected from altitudes above 125 km where the return signals are strongly attenuated first.[27] Rays launched at shallower angles may be reflected from the upper stratosphere at approximately 45 km above the surface in mid latitudes,[27] or from 60–70 km in low latitudes.[17]

Atmospheric scientists have used these effects for inverse remote sensing of the upper atmosphere using microbaroms.[23][28][29][30] Measuring the trace velocity of the reflected microbarom signal at the surface gives the propagation velocity at the reflection height, as long as the assumption that the speed of sound only varies along the vertical, and not over the horizontal, is valid.[27] If the temperature at the reflection height can be estimated with sufficient precision, the speed of sound can be determined and subtracted from the trace velocity, giving the upper level wind speed.[27] One advantage of this method is the ability to measure continuously – other methods that can only take instantaneous measurements may have their results distorted by short-term effects.[8]

Additional atmospheric information can be deduced from microbarom amplitude if the source intensity is known. Microbaroms are produced by upward directed energy transmitted from the ocean surface through the atmosphere. The downward directed energy is transmitted through the ocean to the sea floor, where it is coupled to the Earth's crust and transmitted as microseisms with the same frequency spectrum.[8] However, unlike microbaroms, where the near vertical rays are not returned to the surface, only the near vertical rays in the ocean are coupled to the sea floor.[26] By monitoring the amplitude of received microseisms from the same source using seismographs, information on the source amplitude can be derived. Because the solid earth provides a fixed reference frame,[31] the transit time of the microseisms from the source is constant, and this provides a control for the variable transit time of the microbaroms through the moving atmosphere.[8] 

\section{Balloons}
In the new study, Bowman and his colleagues contributed infrasound payloads to the NASA High Altitude Student Platform (HASP), a yearly program that gives student teams the opportunity to perform experiments on long-duration flights in the stratosphere. During the 2014-2015 flights over Arizona and New Mexico, in which HASP balloons were outfitted with microphones, they detected microbarom signals in the stratosphere for the first time, Bowman said.

The study's authors compared microbaroms detected by their stratospheric sensors to signals from ground-based sensors. They found the stratospheric sensors could detect additional microbaroms and picked up less background noise than ground sensors. While the new study only examined the recordings from a handful of flights, the results indicate balloon-borne sensors are a promising method for detecting other infrasound, like those from natural disasters or nuclear explosions, Bowman said.

The detectors could be used to monitor infrasound generated by nuclear weapons and could help enforce nuclear weapons bans, Bowman said. Balloon-borne infrasound sensors could also be used to detect infrasound in a gaseous planet's atmosphere that could help scientists learn about that planet's interior and phenomena in the atmosphere such as meteor strikes and thunder, Bowman said.

Further research is needed to improve the airborne sensors, Bowman said. Researchers must carefully choose the altitude and time of year for the balloon's flight to help ensure it travels over the desired area. Because the detectors move with the wind, researchers can only tell whether the sound is coming from above or below the sensor, and can't determine the exact direction an infrasound is coming from.

\section{Balistokardiografie}
At every heartbeat, the blood travelling along the vascular tree produces changes in the body center of mass. Body micromovements are then produced by the recoil forces to maintain the overall momentum. The BCG is the recording of these movements, can be measured as a displacement, velocity, or acceleration signal, and is known to include movements in all three axes. The longitudinal BCG is a measure of the head-to-foot deflections of the body, while the transverse BCG represents antero–posterior (or dorso–ventral) vibrations. The original bed- and table-based BCG systems focused on longitudinal BCG measurements, representing what was supposed to be the largest projection of the 3-D forces resulting from cardiac ejection [1]. Table I summarizes modern BCG measurement systems and their axes of measurement. Note that for some systems, head-to-foot and dorso–ventral forces are unavoidably, mixed together in the measurement, and this should be accounted for when interpreting results. However, in spite of the 3-D nature of the BCG, for a long period of time only the microdisplacements of the body along the longitudinal axis (head-to-foot) were considered. Currently, BCG is mainly measured using a force plate or force sensor placed on a weighing scale or under the seat of a chair, with the subject in a vertical position.
\begin{thebibliography}{Mm99}
    \bibitem{microbaroms} Wikipedia. “Microbarom” 4. 12. 2018. \url{https://en.wikipedia.org/wiki/Microbarom}
    \bibitem{balloons} Madeleine Jepsen. “Low-frequency sea sounds ring clear at high altitudes” 4. 12. 2018. \url{https://phys.org/news/2017-09-low-frequency-sea-high-altitudes.html}
    \bibitem{balistokardiografie} Ballistocardiography and Seismocardiography: A Review of Recent Advances. 4. 12. 2018. \url{https://ieeexplore.ieee.org/document/6916998}
\end{thebibliography}
%\pagestyle{empty}
%\section{Přílohy}
%\appendix
\end{document}





